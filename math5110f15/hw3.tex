\documentclass[11pt,reqno]{amsart}
\usepackage[margin=1.1in]{geometry}

\newcommand{\RR}{\mathbb{R}}
\newcommand{\CC}{\mathbb{C}}

\newcommand{\spacer}{\vspace{.4cm}}

\setlength{\parindent}{0cm}

\newcommand{\dist}{\mathop\mathrm{dist}\nolimits}
\newcommand{\diam}{\mathop\mathrm{diam}\nolimits}
\newcommand{\CLB}{\mathop\mathrm{CLB}\nolimits}
\newcommand{\HD}{\mathop\mathrm{HD}\nolimits}

\newtheorem{theorem}{Theorem}

\begin{document}

\begin{center}
\hrule \ \\
\large \textsf{Math 4110/5110, Introduction to Modern Analysis, Fall 2015} \hfill
\Large Homework \#3 \vspace{.25cm} \hrule \
\end{center}

\spacer

\textbf{Due In Class:} Thursday, October 1  \spacer

\textbf{Reading:} Finish reading Chapter 2.  \spacer

Do the following problems. \spacer

\textbf{Problem A:} Let $X$ be a metric space. Let $\mathcal{P}=\mathcal{P}(X)$ denote the collection of subsets of $X$; let $\mathcal{S}=\mathcal{S}(X)$ denote the collection of nonempty subsets of $\RR^2$. Define a function $\mathrm{ex}:\mathcal{P}\times \mathcal{S}\rightarrow[0,\infty]$ by the rule $$\mathrm{ex}(A,B) = \left\{\begin{array}{ll} \sup_{a\in A}\left(\inf_{b\in B}\dist(a,b)\right) &\text{if }A\in \mathcal{S} \\ 0 &\text{if }A=\emptyset.\end{array}\right.$$ (Here we formally write $\sup E = \infty$ for a set $E\subseteq\RR$ if $E$ is not bounded above. By convention $x+\infty=\infty+\infty=\infty$ for all $x\in\RR$.) The quantity $\mathrm{ex}(A,B)$ is called the \emph{excess of $A$ over $B$}.

\quad Prove that excess satisfies the triangle inequality: $\mathrm{ex}(A,C) \leq \mathrm{ex}(A,B)+\mathrm{ex}(B,C)$ for all sets $A,B,C\in\mathcal{P}$ such that $\mathrm{ex}(A,B)$, $\mathrm{ex}(A,C)$ and $\mathrm{ex}(B,C)$ are all defined. \spacer

\textbf{Problem B:} Let $X$ be a metric space. Let $\mathcal{CB}=\mathcal{CB}(X)$ denote the collection of closed, bounded subsets of $X$. Define $\mathrm{HD}:\mathcal{CB}\times\mathcal{CB}\rightarrow [0,\infty)$ by the rule $$\mathrm{HD}(A,B)=\max\left\{\mathrm{ex}(A,B),\mathrm{ex}(B,A)\right\}\quad\text{for all }A,B\in\mathcal{CB}.$$ The quantity $\mathrm{HD}(A,B)$ is called the \emph{Hausdorff distance between $A$ and $B$}.

\quad (a) Prove that $(\mathcal{CB},\mathrm{HD})$ is a metric space.

\quad  (b) Describe in words and/or pictures the open ball $B([0,1]\times\{0\},\frac{1}{4})$ in this metric space $\mathcal{CB}(\RR^2)$.  \spacer

\textbf{Problem C:} Exercise 2.12. \spacer

\textbf{Problem D:} Exercise 2.13. \emph{You must prove your assertions.} \spacer

\textbf{Problem E:} Exercise 2.15. \spacer

\textbf{Problem F:} Exercise 2.22.

\end{document}
